\documentclass[12pt]{article}
\usepackage[paper=a4paper,left=1.5cm, right=1.5cm, top=1.5cm]{geometry}
\usepackage[utf8]{inputenc}
\usepackage{graphicx}
\setlength{\parindent}{4em}
\setlength{\parskip}{1em}
\usepackage{graphicx}

\title{\textbf{THE ROLE OF NUCLEAR MEDICINE IN RESPONSE TO INFECTION IN THE BODY}}
\author{Mayank Mishra\\ Roll No - 18111034 \\ \textit{Semester-8} \\ National Institute of Technology, Raipur }
\date{04\textsuperscript{th} April 2022}

\begin{document}

\maketitle

\section*{Abstract}
\textbf{ Nuclear medicine imaging of infection has two major indications: (a) the localization of a focus of infection in patients with fever of unknown origin; in this context, sensitivity is more important than specificity because subsequent biopsy or morphologically based imaging can be performed; and (b) the diagnosis of an infection in patients with localised symptoms, for example after surgery, when normal anatomy is absent or when morphologically based imaging cannot be performed. High sensitivity and, to a lesser extent, high specificity are required in these instances to guide subsequent clinical therapy (conservative or surgical). Technetium-99m nanocolloids, gallium-67 citrate, indium-111- and 99mTc-labeled white blood cells, 99mTc-antigranulocyte antibodies, and 99mTc-or llin-labeled unspecific human IgG all have diverse biodistributions and physical properties.The lack of physiological absorption in an organ and a patient's radiation exposure are grounds to utilise different radiopharmaceuticals in different therapeutic contexts, each tailored to the patient's unique circumstances.}




\section*{Introduction}
 Nuclear medicine imaging gives information on changes in the patient's pathophysiological and pathobiochemical processes, whereas morphologically based imaging provides high-resolution information on the morphological changes that occur in a particular process.
\\In ordinary clinical practise, nuclear medicine imaging is accessible as wholebody imaging, whereas computed tomography (CT), magnetic resonance imaging (MRI), and other modalities give information on a single body area.
\\A nuclear medicine method with great sensitivity in all parts of the body is required for the localisation of an infectious condition. Fever of unclear cause is a good example.
\\Following the diagnosis of an infection or inflammation, further tests such as CT, MRI, biopsy, and culture are required. Infection-specific approaches are required for the differential diagnosis of inflammation or infection, such as after a surgical operation when the surgeon has to know whether to reopcrate or offer conservative therapy.





\section*{PATHOPHYSIOLOGICAL ASPECTS OF IMAGING INFECTION }
To display all of the common events in infectious disorders, all nuclear medicine techniques employ one single phase of the infection cascade as a part of the host defence mechanism. Non-specific and specific host defence mechanisms can be distinguished. Specific responses are aimed against a single pathogen, whereas non-specific responses guard against a wide range of infectious organisms.[1]
\\With a radiolabelled monoclonal antibody against Pneumocystis carinii, precise infection imaging focused against one microbe has been established, with a claimed sensitivity of 85.7 percent and a specificity of 86.7 percent. However, this method can only be employed in a patient group where there is a significant possibility of only one type of infectious agent being present.
\\Physical barriers to invasion, secretions at possible entrance portals, complement, and phagocytic cells (granulocytes and monocytes) are examples of non-specific defence mechanisms. Monocytes are also an important part of the immune response mediated by particular cells. PMNs (polymorphonuclear neutrophils) are important in the host's defence against bacteria, viruses, and fungi, but they can cause tissue damage in several non-infectious disorders. is the ingestion and killing of invading pathogens, a multi-step process. Adherence to endothelium surfaces is one of the initial steps in the neutrophil's response to inflammatory stimuli, which is mediated by cell surface glycoproteins (non-specific cross-reacting antigens) and adhesion molecules produced on the epithelial side. In response to chemotactic chemicals produced by invading bacteria, phagocytic cells at the implicated location, or activation of the classical complement pathway, neutrophils move to sites of infection or inflammation. Neutrophils detect these compounds in nanomolar amounts and travel in a directed manner towards them. Increased blood flow and capillary permeability precede this migration of granulocytes out of the circulation, resulting in the well-known signs of infection, such as calor, rubor, tumour, and dolor (Celsus, 30 B.C.-50 A.D.), as well as reduced function of the infected part of the body (Celsus, 30 B.C.-50 A.D.) (functio laesa; Galen, 131-200 A.D.).The goal of an organism's inflammatory response is to eradicate the inciting substances.
\\Hyperaemia, enhanced vascular permeability, protein exudation, and leucocyte migration are all signs of acute inflammation. Macrophages, lymphocytes, and plasma cells predominate in chronic inflammation, but hyperaemia and vascular permeability are less severe. Infections and inflammatory disorders both have these pathophysiological pathways. There is now no commonly accessible method that can answer the surgeon's most essential question, which is whether there is merely inflammation in a wound after surgery, allowing conservative care, or whether there is bacterial infection, requiring rapid surgical intervention. Although this issue cannot be answered with commonly accessible radiopharmaceuticals, studies utilising technetium-99mlabelled chinolons (infecton) have shown infections before labelled granulocytes travel to foci.[2]
\begin{figure}[htp]
    \centering
    \includegraphics[width=10cm]{nuc.JPG}
    \caption{\textit{ Possible targeting with nuclear medicine procedures in patients with infectious and inflammatory diseases.}}
    \label{Figure}
\end{figure}





\section*{RADIOPHARMACEUTICALS FOR IMAGING INFECTION }
Only those radiopharmaceuticals that are usually accessible will be discussed here, out of all those documented in the literature for the localisation of infectious illnesses. Gallium-67 binds to lactoferrin receptors, although it must likely first be taken up unspecifically in the focus. The same absorption processes are explored for 99mTc-labeled nanocolloids as well as 99mTc- and indium-11 l-labeled human immunoglobulin (HIG), the latter of which is currently unavailable commercially. Increased capillary permeability, which occurs just before leucocytic migration and is a symptom of infection but also inflammation, is the main method by which these radiopharmaceuticals are absorbed.
\\The chemotactically driven migration of leucocytes is clearly demonstrated in vitro . Monoclonal antigranulocyte antibodies attach to circulating granulocytes in part and circulate as free antibodies in part, ready to target granulocytes at the focus after passing through the capillary wall unspecifically.[3]
\\The decision to undertake a specific infection scan must always be based on the clinical condition; nevertheless, the radiopharmaceutical's biodistribution and radiation dose also play a role. The considerable radiation dose  associated with 67Ga citrate or 11qn-labeled leucocyte examinations restricts their usage to select, clinically significant conditions. In situations of probable CMV pneumonia in AIDS-compromised individuals or chronic vertebral osteomyelitis, 67Ga citrate is utilised, and l lqn-labelled leucocytes are used if renal infection must be excluded or proved. Because 24-hour scans are feasible, kidney infections and their distinction from abscesses and enteric communications are best investigated using 11 lin labelled leucocytes. The intestinal excretion of 99mTc-hexamethylpropylene amine oxime (HMPAO) tagged leucocytes, on the other hand, indicates that the colon is clear of activity within 3 hours post-injection. Only tracers with a very low activity in the circulation, such as in vitro labelled leucocytes (111In or 99mTc) or antigranulocyte antibodies, should be injected if the goal of the study is to prove vascular prosthetic or heart valve infection; tracers with a very high activity in the circulation, such as 99m Tc-HIG, show only limited tracer uptake in cases of moderate infection.
\section*{67Ga citrate}
\\\textbf{Physics}: 67Ga citrate is a cyclotron-generated radiopharmaceutical that emits gamma rays with prominent energy of 93, 184, 296 and 388 keV throughout a broad range of 93-880 keV. Physically, the half-life is 78 hours.
\\\textbf{Biodistribution}: Over the first 24 hours, 10 percent to 25 percent of the nuclide is eliminated via the kidneys, but the colon is the primary route of excretion after that. About 75$\%$ of the injected amount is still in the body 48 hours later, evenly divided across the liver, bone and bone marrow, and soft tissues[5]. This even distribution in several organs predetermines the need for imaging. Other radiopharmaceuticals may be more effective in examining organs with physiological absorption of 67Ga citrate.
\\\textbf{Uptake mechanism}: Some intracellular lactoferrin is excreted by leucocytes at the site of inflammation or infection, but it stays localised and attached to macrophages.67Ga citrate is delivered either ionically or attached to transferrin, and it can seep past the vascular epithelium at the infection site, where it is bound to lactoferrin. Another method might be the creation of siderophores with binding affinity for both iron and gallium by microorganisms cultured in a low iron environment: in the absence of readily accessible iron, 67Ga could be carried straight into the cell.
\\\textbf{Clinical Indications}: The use of 67Ga is limited to certain indications such as fever of unknown origin, chronic osteomyelitis of the spine, and lung infections, especially in immunocompromised patients, due to high radiation exposure, lack of availability, and unfavourable physical characteristics for gamma camera imaging. These features, combined with the availability of 99mTclabelled radiopharmaceuticals for imaging infections, limit the use of 67Ga to certain indications such as fever of unknown origin, chronic osteomyelitis. The diagnostic "fever of unknown origin" (FUO) is reserved for individuals who have had a high temperature (38.8 ° C) for at least two weeks and whose source of fever has remained unexplained after extensive examinations . These strict criteria exclude out evident bacterial and viral illnesses, as well as those in which the diagnosis becomes clearer with time and those whose fever is caused by a non-infectious aetiology. It does not, however, rule out all individuals with a persistent fever who are suffering from an infectious condition. The average percentage of FUO patients having an infectious aetiology was determined to be 32.7 percent[6-7] in many studies that looked at a total of 647 people with the disease. Collagen vascular disorders and neoplasms are other common causes of FUO. In the diagnosis of infections, it is widely known that 67Ga has a high sensitivity but a low specificity.
\\This is required in FUO patients because a lesion must first be recognised before other diagnostic procedures, such as biopsy, may be used. The poor specificity also allows for the discovery of neoplasms and, in certain cases, collagen vascular disorders, which may then be investigated further using more specific procedures.
\\In patients with suspected vertebral osteomyelitis, the outcomes of white blood cell imaging , immunoscintigraphy , and 99mTc-HIG imaging of infections are not particularly promising. This is difficult to explain, but it's possible that persistent infection causes a very high pressure in one disc, preventing proteins and granulocytes from moving into the focus during the imaging time. Although the data is limited, 67Ga citrate imaging for vertebral osteomyelitis may be superior than labelled leucocyte, antibody, and immunoglobulin imaging, albeit it is less selective and detects tumours as well.
\\Most suspected respiratory infections may be confirmed with a chest X-ray and sputum cultures, which are both readily available.Because white blood cells, antibodies, and immunoglobulins fail to accurately detect pneumonia, radionuclide studies are not widely used in this field. 67Ga has emerged to play a role in the diagnosis of AIDS-related respiratory illness in this setting.
\\67Ga lymph node uptake is a common observation in HIV-positive patients and can be found almost everywhere on the body. This is a common occurrence in people with TB and lymphoma caused by Mycobacterium tuberculosis. In the case of bacterial pneumonia, localised pulmonary uptake is more common, whereas widespread pulmonary uptake can be caused by Pneumocystis carinii pneumonia, CMV pneumonia, interstitial pneumonia, and pneumonitis.[8] A negative scan indicates that an infectious process is not present.
\section{99mTc-nanocolloids}
\\\textbf{Physics}-With a physical half-life of 6 hours and a gamma energy of 40 keV, the physics of 99rnTc as a generator product are well established. Furthermore, it is constantly available in every nuclear medicine unit and results in a low level of patient radiation exposure.
\\\textbf{Biodistribution}-In the rat model, the biodistribution of 99mTc-nanocolloids has been extensively studied. Within the time relevant for imaging, blood activity decreases from 3.4 percent (30 min) to 1.8 percent (1 h) to 0.5 percent (3 h) of the injected dose; in the liver, it decreases from 72.4 percent to 68.7 percent to 52.5 percent; in the bone marrow, the figures are 14.5 percent, 12.5 percent, and 12.4 percent, respectively; thus, this radiopharmaceutical can also be used for bone marrow imaging. Nanocolloids are lysosomally destroyed following active absorption in the reticuloendothelial system, therefore 54.5 percent are renally eliminated after 24 hours.
\\\textbf{Uptake Mechanism}-Approximately 86 percent of these particles have a diameter of 30 nm or less, with the remainder having a diameter of 30 to 80 nm[9]. The enhanced permeability of the capillary basal layer caused by cytokines is responsible for the entry of these inert particles into the pericapillary spaces and their subsequent accumulation. However, unlike macromolecules, blood clearance is quick, allowing imaging to be finished in as little as 2-4 hours. The activity may then arise in the gastrointestinal system, most likely as a result of spontaneous oxidation to pertechnetate. The absolute concentration of 99mTc-nanocolloids was found to be roughly 100 times lower than that of 11in labelled leucocytes, and the abscess/blood concentration ratios for 99mTc-nanocolloids were likewise lower than for other radiopharmaceuticals. However, measurements were taken 24 hours after the radiopharmaceuticals were injected, which is an inappropriate period for nanocolloids; therapeutically relevant times are between 1 and 4 hours after administration.
\\\textbf{Clinical Indications}-Infections and inflammations of the skeleton and joints are the most common indications for nanocolloids.Soft tissue infections, particularly gastrointestinal infections, are difficult to identify, possibly due to a delayed absorption process in soft tissues as well as a natural elimination of 99mTc destroyed particles after 4 hours post-injection.
Only a few studies have looked into 99mTcnanocolloids in individuals with osteomyelitis clinically. The results show values that are similar to those of llIn-labeled leucocytes, with sensitivity ranging from 87$\%$  to 95$\%$  and specificity ranging from 77$\%$  to 100$\%$ .As a result, several researchers have concluded that 111In- and 99mTc-labeled leucocytes are interchangeable . Another study  found that l llIn-labeled leucocytes had a sensitivity of 75$\%$  and 99mTc-nanocolloids had a sensitivity of 94$\%$ , with a specificity of 90$\%$  and 84 percent, respectively.As a result, their diagnosis accuracy was 85 percent in one case and 87 percent in the other.When patients with "slightly enhanced activity" were considered negative in this investigation, the authors computed the sensitivity and specificity. As a result, the specificity was around 100 percent. Because 99mTc-nanocolloids can exit the circulation wherever permeability is raised, regardless of infection or non-specific inflammation, this "manipulation" appears to be essential. The two kinds of lesion cannot be distinguished, which is also true for most other imaging agents. However, the quick localisation of an infectious process within 30-60 minutes, after which the research may be ended, is a significant benefit of these nanocolloids.
Rheumatoid arthritis is also helped by 99mTc-nanocolloids.Liberatore et al. observed that 79 percent of all actively inflammatory joints had a positive scan with 99mTc-nanocolloids, while 95 percent of all clinically negative joints had a negative scintigram in a recently published research. When they compared the findings of 99mTc-nanocolloids and 99mTc-HIG, they discovered that both radiopharmaceuticals agreed with clinical tests; 99mTc-HIG appeared to have certain benefits only in the early stages of the disease.
\section{lllln-labelled autologous leucocytes}
\\\textbf{Physics}-One of the oldest radiopharmaceuticals for imaging infection, 67Ga citrate, has a low target-to-background ratio, requires 48- or 72-hour pictures, has intestinal excretion, and may detect both infection and tumours. llIn-labeled autologous leucocytes have been shown to have a substantially greater specificity. With a physical half-life of 67 hours and photopeaks of 173 and 247 keV, 111In allows for delayed imaging of up to 24 hours and relatively efficient imaging with ordinary gamma cameras.
\\\textbf{Biodistribution}-The quantity of neutrophils that concentrate in sites of inflammation is quite high: up to 10 of all circulating neutrophils might be found there every day.
48.5 percent of the activity was identified in the liver, 11 percent in the spleen, 7.9 percent in the lungs, and 8.5 percent in the blood twenty-four hours following re-injection of I l lIn oxine-labeled cells in dogs . This biodistribution, which is quite similar to that of people, shows that splenic and liver abscesses, as well as lung abscesses, may be difficult to identify. However, because there is no outflow from the kidneys, bladder, or colon, the whole abdomen is a good region for detecting viral or inflammatory disorders. Soft tissue infection in the muscles is also visible.
\\\textbf{Cell isolation and labelling}-111 Lipophilicity is a feature that both l aqn-tropolonate and 99mTc-HMPAO share in oxine. As a result, llIn labels all cell types equally.As a result, before labelling, leucocytes must be isolated from other blood cells. Separation and labelling of cell subsets, such as lymphocytes and monocytes, is also possible. Thakur et al. [10]were the first to describe the labelling process, which was later modified by other scientists. Other lipophilic compounds, such as tropolonate and acetylacetone, can be employed instead of oxine. Although the difference in cell kinetics is not universally acknowledged, tropolonate-labeled cells fared better at early imaging times than oxine-labeled cells.
\\\textbf{Uptake mechanism and physiology}-Neutrophils have a two-week life span. After 6-12 days, mature neutrophils are discharged into the peripheral circulation from stem cells in the bone marrow. About half of the cells are in the circulation pool, which may be evacuated, while the other half are in the marginating pool, where they are temporarily sequestered in capillaries or adhere to the endothelium of bigger veins. After physical exertion, epinephrine injection, and bacterial endotoxin exposure, the cells can migrate between the two pools. Cells move from the marginating to the circulation pool in all situations . Emigration of leucocytes by chemotactic stimuli occurs as early as 30-40 minutes after neutrophil activation in inflammatory or infectious disorders.
\\\textbf{Radiation effects.}- After in-leucocyte labelling, the leucocytes are exposed to two principal sources of radiation.The first is external radiation absorbed by leucocytes during the labelling process and subsequent incubation. Internal radiation, derived from the 1 JIn contained in the cells, is the second and more important source of radiation. The majority of the dosage is obtained internally from low-energy Auger electrons (0.6-25.4 keV), which have a range significantly less than the cell width, resulting in a very high radiation load of up to 14.8 Gy. Nonetheless, there is substantial evidence that the carcinogenic risk to the patient is quite minimal, even in mixed-cell populations of leucocytes marked with 500 aCi of 111in.
\\\textbf{Clinical indications}-While leucocyte scintigraphy is very specific for leucocytic infiltration, which may be found in a variety of disease entities ranging from infection to tumours, it is not specific for bacterial contamination. Despite this, because of their great selectivity for leucocytic infiltration, lSIn-labeled leucocytes offer a wide range of possible applications. Patients with chronic inflammatory bowel illness or other intestinal infections, osteomyelitis, prosthetic valvular heart disease, prosthetic vascular prostheses, renal ailments, lung infections, and fevers of unknown origin have all been thoroughly studied.
Because of the physical properties of 111in, such as its medium energy, long physical half-life, and high radiation burden on patients, as well as its production in a cyclotron, which results in limited availability, the resolution achieved with gamma camera imaging is suboptimal, and 11IIn is not recommended in clinical practise. Given the potential of labelling leucocytes using 99mTc-HMPAO, there are only a few circumstances in which the use of l lIn is required. The physical biodistribution and excretion of 99mTc-HMPAO leucocytes prevents the diagnosis of infection in the kidneys, bladder or small pelvis, gallbladder, and gut, where 4-h and 24-h images are required for the differentiation of segmental inflammation or abscess due to physical 99mTc excretion via these organs. llln-labeled granulocytes can be substituted with 99mTc-HMPAO leucocytes in all of the other conditions listed above.
\section{99mTc-HMPAO labelled leucocytes}
\\There has been an obvious need for a 99mTc labelling process for leucocytes because of the cost, simplicity, picture quality, and dosimetry.
\\\textbf{Physics}-99mWc's physics are well-known. The energy of 140 keV, the physical half-life of 6 h, the availability in any nuclear medicine centre, and the cost are all positives. The 99mTc-HMPAO labelling of leucocytes is the only reliable and repeatable approach. The lipophilicity of HMPAO is the basic mechanism of 99mTc-HMPAO labelling, which permits the 99mTc to enter the cells, which must also be separated before labelling.
\\\textbf{Biodistribution and clinical indications}. 99mTc-HMPAO is not extremely stable in all labelled cell types, whereas t l 1 In is. In vitro, we found that up to 7$\%$ of the label was eluted from tagged leucocytes every hour. Due to the instability of 99myc, bound secondary hydrophilic complexes of HMPAO are eliminated via the kidney and bladder few minutes after the tagged cells are re-injected. Gallbladder visualisation begins 2 to 3 hours after re-injection, while small bowel visibility begins 3 hours after re-injection. These excretion routes establish the indications for 99mTc-HMPAO, which include infections other than those of the kidney, bladder, and gallbladder, inflammatory bowel illness, or abscesses with enteric communications, as well as the distinction between abscesses and segmental inflammation. 30 minutes after re-injecting the cells, chronic inflammatory bowel illness can be noticed. As a result, defective segments can be identified before physiological excretion begins.
\\Several studies have shown that 99mTc is superior to llln in the treatment of inflammatory bowel illness . This is due to its greater capacity to pinpoint illness to particular bowel segments as well as its ability to accurately detect small bowel disease. In chronic inflammatory bowel disease, the role of labelled white blood cells is to localise diseased bowel segments in highly acute disease with the risk of perforation during endoscopy and in chronic disease with stenosis where endoscopy cannot pass the stenosis to localise the total number of diseased segments.
\\Today, 99mTc-HMPAO labelled cells are also recommended for the identification of complications such as fistulas and abscesses, but only llln labelled cells are helpful for quantifying disease activity. In patients with osteomyelitis, the sensitivity of 99mTc-HMPAO tagged leucocytes is equivalent to that of 111In labelled leucocytes, according to the author's experience. Despite its greater resolution, 99mTc-HMPAO does not appear to be more accurate since imaging of chronic osteomyelitis is more common, necessitating late scans due to low granulocyte turnover. Furthermore, 99mTc-HMPAO tagged leucocytes cannot solve the problem of spinal osteomyelitis, which is characterised by a lower cell uptake. It's not clear why infection in the vertebrae is more difficult to confirm using tagged leucocytes and antibodies than infection in other parts of the axial skeleton with similar bone marrow. It's possible to speculate that the high pressure in one diseased vertebra limits granulocyte uptake, or that a vertebra's "coldness" is just relative owing to the physiological bone marrow in the neighbouring vertebral bodies.
\\In the treatment of patients with intrathoracic disease, labelled leucocyte scanning plays a small role. According to certain findings, tagged leucocytes can help patients with bronchiectatic lobes and patients with vasculitis before surgery.
\\Because of the increased stability in both circulating cells and targeted illness, as well as the longer physical halflife, evaluation with l lqn-labelled leucocytes is probably preferred in patients with fever of uncertain aetiology.However, clinical proof is still pending.
\section{99mTc-labelled antigranulocyte antibodies}
\\Immunoscintigraphy has an advantage over autologous leucocyte methods in the imaging of infection because it is easier to employ than techniques that involve the isolation of autologous white blood cells.
\\\textbf{Biodistribution and uptake mechanism.} Antigranulocyte antibodies that are 99mTc-labeled are directed against the non-specific cross-reacting antigen 95. (NCA-95). Locher et al. were the first to disclose the use of a 123I-NCA-95 immunoglobulin (IgGl) against granulocytes; shortly after, Joseph et al. reported the use of a 99mTc-labeled anti-NCA-95 antibody, which is today the most extensively used antibody. Becker et al.  recently described the use of a 99mTc-labeled anti-NCA-90 Fab' fragment, however this promising fragment is not yet widely accessible. Only granulocytes, promyelocytes, and myelocytes express the antibody-targeted epitope in humans.2x109M was estimated as the affinity constant of the 99mTc-antigranulocyte antibody (IgG1) BW 250/183. Despite its high binding capacity, BW 250/183 has little effect on granulocyte-mediated activities. Around 10$\%$ to 20$\%$ of the radiolabeled entire antibody injected is linked to circulating granulocytes that are functionally normal and may target an infected region. Around 19$\%$ of the injected antibody circulates as free immunoglobulin, with the same potential as non-specific human IgG.This frequently causes issues in ordinary clinical practise, as inflammatory lesions are scanned as well owing to increased capillary permeability, making it difficult to distinguish between inflammatory non-specific and granulocyte-associated specific uptake in a lesion.
\\The following is a summary of the current hypothesis for the absorption process of radiolabelled monoclonal antigranulocyte full antibody: (a) nonspecific, non-antigen-related uptake of free antibody due to increased capillary permeability at the focus, with subsequent binding to granulocytes; and (b) nonspecific, non-antigen-related uptake of free antibody due to increased capillary permeability at the focus, with subsequent binding to granulocytes. The high target-to-background ratios that arise are owing to the precise binding of large numbers of injected antibodies to epitopes in bone marrow and spleen, as well as the associated low background activity. With this fast endogenous background removal, tiny lesions may be detected with great picture quality.
\\\textbf{Clinical Indications}-Antigranulocyte antibody scintigraphy had a sensitivity of 69 percent for the hips, 79 percent for the thigh, 85 percent for the knees, and 100 percent for the lower leg and ankle in 106 patients with bone infection. The sensitivity of the bone reduced as it moved from the periphery to the centre. This might be owing to the antibody's physiological uptake in the bone marrow-containing regions of the bone, where normal antibody uptake makes it difficult to differentiate normal bone marrow from tiny infected foci. Other theories propose that in times of infection, increasing pressure in the vertebral bodies inhibits tagged white cells or antibodies from entering.
Haematomas, contusions, and aseptic inflammation can all cause false-positive uptake. In chronic inflammatory bowel disease, 49 percent of segments could be identified by 99mTc- BW250/183 scintigraphy at 2 h p.i., 55 percent at 5 h p.i., and 91 percent at 20 h p.i. Segarra et al. showed sensitivity, specificity, and accuracy to be 61 percent, 100 percent, and 78 percent at 4 h and 79 percent, 92 percent, and 78 percent at 24 h p.i. in another prospective trial of CIBD. It should be noted that the activity does not pass through the intestinal wall, as evidenced by the use of 111In-labeled white blood cells.Over the next 24 hours, the activity in the gut wall rises.
\\Following a clinical diagnosis of vascular graft infection, nuclear medicine methods are employed to identify the full extent of the illness. Because of the quick endogenous background subtraction and minimal vascular activity, immunoscintigraphy with specific antigranulocyte antibodies is the best method for locating vascular graft infections. The sensitivity is said to be 94$\%$ and the specificity is said to be 85$\%$. In perivascular haematoma, falsepositive results have been seen, particularly in late 24-hour imaging; such findings must be ruled out by ultrasonography and computed tomography.
\\The combination of immunoscintigraphy with SPET and echocardiography detects subacute infective endocarditis with great sensitivity in prosthetic vascular heart disease.
\\The use of antigranulocyte antibodies in immunoscintigraphy in patients with pneumonia or other infective lung disorders has yet to be thoroughly investigated.We discovered that lung abscesses are easily detectable, but we were unable to identify pneumonia. Another study on false-negative imaging in lung abscesses supports our findings. Similar findings in AIDS patients reveal that extra-abdominal infections can be eliminated or demonstrated, but opportunistic pneumonia is never demonstrated.
\\99mTc-labeled antibodies are widely used because to their ease of use and excellent sensitivity in detecting a target in the body. However, in patients in which the surgeon has to distinguish between postsurgical inflammation and infection, the scans must be carefully interpreted because both circumstances provide a signal. Despite the fact that the antibodies are routinely used in clinical diagnostics, they induce antimouse antibodies in humans (HAMAs). The incidence of such antibodies appears to be dosage dependant, ranging from more than 30$\%$ in individuals getting recurrent injections to 4.5 percent in patients receiving a fixed dose of the antibody of 125 lag. As a result, we propose that the antibody be used with no more than 250 latency to guarantee a low HAMA response rate.
\section{CONCLUSION AND FUTURE PROSPECTS}
For imaging infection, a variety of radionuclides are available.They differ in terms of physical properties, biodistribution, cell isolation requirements, and radiation exposure. Based on all of the information available in the literature on all of the benefits and drawbacks of commercially (i.e. widely) accessible radiopharmaceuticals. None of the approaches presented can answer the surgeon's most critical question in a patient with clinical symptoms of inflammation or infection following surgery: whether the signs are suggestive of infection or just inflammation. While inflammation can be treated conservatively, infection necessitates further surgical intervention. Nonetheless, contemporary technologies allow for some distinction; for example, if the absorption of a radiopharmaceutical around a knee prosthesis is intense but homogenous, it is most likely suggestive of inflammation, but if it is more focused, it is most likely indicative of infection. As a general rule, good scan interpretation is possible when all of the relevant anamnestic and clinical data is taken into account.
\section*{References}
\begin{itemize}
\item [1].Woods G, Gutierrez Y, Walker D, Purtilo D, Shanley J. Diagnostic pathology of infectious diseases. Philadelphia: Lea &Febiger, 1993
\item [2].Bomanji J, Solanki KK, Britton KE, Siraj Q, Small M. Imaging infection with Tc-99m-radiolabelled "infecton". Eur J Nucl Med 1993; 20: 834.
\item [3].Radiation dose to the patients from radiopharmaceuticals. Gallium citrate (ICRP publication no. 53). Ann ICRP 1987;
\item [4].Radiation dose to patients from radiopharmaceuticals. Technetium labelled white blood cell (leucocytes) (ICRP publication no. 53). Ann ICRP 1987; 18: 231-232.
\item [5].Hoffer R Gallium: Mechanisms. J Nucl Med 1980; 21:282-285.
\item [6].Howard P Jr, Hahn HH, Palmer PL. Fever of unknown origin.A prospective study of 100 patients. Tex Med 1977; 73: 56.
\item [7].Knockaert DC, Vanneste LJ, Vanneste SB. Fever of unknown origin in the 1980s. An update of the diagnostic spectrum.Arch Intern Meal 1992; 152: 51.
\item [8].Palestro CJ. The current role of gallium imaging in infection.Semin Nucl Med 1994; XXIV: 128-141.
\item [9].De Schrijver M, Streule K, Senekowitsch R, Fridrich R. Scintigraphy of inflammation with nanometer-sized colloidal tracers.Nucl Med Commun 1987; 8: 895-908.
\item [10].Thakur ML, Lavender JR Arnot RN, Silvester DJ, Segal AW.Indium-l 1 l-labeled autologous leukocytes in man. J NucliVied 1977; 18: 1012-1021.


\end{itemize}






 \end{document}
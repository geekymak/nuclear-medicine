\documentclass{article}
\usepackage[utf8]{inputenc}

\title{update 8}
\author{Mayank Mishra }
\date{March 2022}

\begin{document}

\maketitle

\section{discussion}
In vascular graft infection, nuclear medicine procedures
are used to determine the complete extent of the
infection following its clinical diagnosis. Immunoscintigraphy
with specific antigranulocyte antibodies offers
the best means for the localization of vascular graft infections
due to the rapid endogenous background subtraction
and low vascular activity. The sensitivity is reported
to be 94% and the specificity 85% [81]. Falsepositive
results have been observed in perivascular haematoma,
especially in late 24-h images; such findings
have to be excluded by ultrasonography and computed
tomography.
In prosthetic
\end{document}

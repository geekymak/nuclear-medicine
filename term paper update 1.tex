\documentclass{article}
\usepackage[utf8]{inputenc}

\title{The role of nuclear medicine in response to infection in the body}
\author{MAYANK MISHRA }
\date{January 2022}

\begin{document}
\maketitle

\section{Introduction}
Nuclear medicine imaging of infection has two major indications: (a) the localization of a focus
of infection in patients with fever of unknown origin; in this context the radio-pharmaceutical
should be highly sensitive whereas specificity is not so important because subsequent biopsy or
morphologically based imaging can be performed; (b) the diagnosis of an infection in patients
with localized symptoms, for example after surgery, when normal anatomy is absent or when
metal implants prevent computed tomography or magnetic resonance imaging. In these latter
cases high sensitivity and to an even greater extent high specificity are mandatory to guide
further clinical management (conservative or surgical). All radiopharmaceuticals available to
date, such as technetium-99m nanocolloids, gallium-67 citrate, indium-111- and 99mTc-labelled
white blood cells, 99mTc-antigranulocyte antibodies, and 99mTc-or l llin_labelled unspecific
human immunoglobulin, have different biodistributions and different physical characteristics.
The absence of physiological uptake in an organ and the radiation exposure of a patient are
reasons to use different radiopharmaceuticals in different clinical situations, adapted to the
inidividual carcumstances of the patient.
1
\end{document}
